% !TeX spellcheck = en_US
%%%%%%%%%%%%%%%%%%%%%%%%%%%%%%%%%%%%%%%%%%%
%
% szablon pracy licencjackiej 
% korzystający ze stylu pracalicmgr.cls
% 2017.03.01 P. Durka durka@fuw.edu.pl 
% na podstawie pliku J. Żygierewicz 2016
%
%%%%%%%%%%%%%%%%%%%%%%%%%%%%%%%%%%%%%%%%%%%



\documentclass{pracalicmgr}
\usepackage{polski}
\usepackage[utf8]{inputenc}

\usepackage{graphicx}
\usepackage{multicol}
\usepackage{caption}
\usepackage{subcaption}


\usepackage{float}
\graphicspath{ {./images/} }

%\usepackage[round]{natbib}   % omit 'round' option if you prefer square brackets
%\bibliographystyle{plainnat}
%\usepackage[maxcitenames=2, backend=biber, style=authoryear]{biblatex}
%\renewcommand*{\nameyeardelim}{\addcomma\space}
%\renewcommand*{\nameyeardelim}{\addcomma\space}

\usepackage{ gensymb }

%@@ -0,0 +1,8 @@
\renewcommand{\contentsname}{Table of contents}
\renewcommand{\listtablename}{List of tables}
\renewcommand{\listfigurename}{List of figures}
\renewcommand{\chaptername}{Chapter}
\renewcommand{\tablename}{Table}
\renewcommand{\figurename}{Figure}
\renewcommand{\appendixname}{Appendix}
\renewcommand{\abstractname}{Summary}


\author{Alicja Krześniak}

\nralbumu{335244}


\title{Emisje otoakustyczne wywołane przez bodźce tonalne w porównaniu z emisjami produktów zniekształceń nieliniowych}

\tytulang{Comparison of Single Frequency Otoacoustic Emissions and Distortion Product Otoacoustic Emissions}

\kierunek{Zastosowania Fizyki w Biologii i Medycynie}

\specjalnosc{Neuroinformatyka}

\opiekun{dr hab. Jarosława Żygierewicza 
	\\Zakład Fizyki Biomedycznej
	\\Instytut Fizyki Doświadczalnej
	\\Wydział Fizyki, Uniwersytet Warszawski
	\\oraz
	\\prof. nadzw. dr hab. n o zdr. Wiesława Jędrzejczaka
	\\Zakład Audiologii Eksperymentalnej
	\\Instytut Fizjologii i Patologii Słuchu 
	\\Światowe Centrum Słuchu}

%\dziedzina{13.200}
\dziedzina{13.2 Physics}

\date{czerwiec 2018}

\keywords{Otoacoustic Emissions, DPOAE, SFOAE}

\bibliography{bibliografia}

\begin{document}


    \maketitle
    \let\cleardoublepage\clearpage
    
    \begin{abstract}
Krótkie (maks. 800 znaków):

Comparison of Single Frequency Otoacoustic Emissions (SFOAE) and Distortion Product Otoacoustic Emissions (DPOAE), concerning their spójność i powtarzalność. 

     \end{abstract}

  
    \tableofcontents
    
    \chapter*{Goal}
    \addcontentsline{toc}{chapter}{Goal}
    The aim of this work was to check whether Single Frequency Otoacoustic Emissions are a clinically useful and robust procedure, which is better than currently used Distortion Product Otoacoustic Emmissions.
    
    \chapter{Introduction}
    Problemy ze słuchem mogą mieć podłoże neurologiczne lub mechaniczne związane z uszkodzeniem narządów słuchu. W tej pracy zajmę się tylko kwestiami związanymi z drugim przypadkiem.
    \section{Ear anatomy}
    %    \begin{figure}[htbp]
    %    	\begin{center}
    %    		
    %    	\end{center}
    %    	\caption{Inner ear anatomy (modified from \cite{}).}
    %    	\label{rys:inner_ear}
    %    \end{figure}
    \section{Hearing check procedures}
    \subsection{Impedance audiometry and Tympanometry}
    \subsection{Tonal Audiometry}
    \subsection{Objective methods}
    For screening, small children and newborns we need an objective and robust test which doesn't depend on their cooperation.
      

   \section{Otoacoustic Emmissions}
   \subsection{Brief history}
   \subsection{Mechanism}
   \subsection{Types}
   \subsubsection{Spontaneous}
   \subsubsection{Click and Transient}
   \subsubsection{Distortion Product}
   \subsubsection{Single Frequency}

%	\begin{figure}[H]
%	\begin{subfigure}{.33\textwidth}
%		\centering
%
%		\caption{Magnocellular}
%		\label{rys:magno}
%	\end{subfigure}%
%	\begin{subfigure}{.33\textwidth}
%		\centering
%
%		\caption{Parvocellular}
%		\label{rys:parvo}
%	\end{subfigure}

	
 
\chapter{Materials and methods}   
 	All experimental procedures  were accepted by ...komisja etyczna... with written consent signed by every subject.
 	\section{Subjects}
 	 The data was gathered from 20 subjects ( ... males, ... females), aged ....-.... . There was a group of ... subjects with healthy ears and both types of emmissions present. For ... subjects only DPOAE were present and ... subjects were rejected from further tests and analysis after the preliminary hearing check procedures.
 	
 	
 	\section{Preliminary hearing check procedures}
The experiments took place in World Hearing Center in Kajetany with great help from mgr inż. Edyta Piłka.
		\subsection{Check of the inner ear status}
		It contained impedance audiometry and tympanometry measured with ... Haering Reflex Threshold and audiogram type A.
		\subsection{Tonal audiometry}
		Conducted in soundproof cabin for frequencies in range... with step ... Hz. No bone conductance.
 	
 	\section{Otoacoustic emissions recording}
 	The measurements were made in World Hearing Center in Kajetany and in the Faculty of Physics at the University of Warsaw using Mimosa Acoustic system. 
 	\subsection{Overall experiment scheme (!poprawić sformułowanie!)}
 	     %    \begin{figure}[htbp]
 	     %    	\begin{center}
 	     %    		
 	     %    	\end{center}
 	     %    	\caption{Experimental procedure}
 	     %    	\label{rys:experiment}
 	     %    \end{figure}
	\subsection{DPOAE}
	\subsection{SFOAE}
	Frequencies with low Signal to Noise Ratio (SNR) were retested once to maximise number of frequencies categorised as "passed". For repeating "Stimulus level out of range" warning, the recalibration was performed.
	\subsubsection{Quick}
	\subsubsection{Cluster}
	\subsubsection{Spectrum}
	The longest ..procedure.. (it takes approximately ... minutes for one measurement) testing 60 frequencies showed on %Fig. \ref{rys:long_spectrum}  
	 	     %    \begin{figure}[htbp]
	 	     %    	\begin{center}
	 	     %    		
	 	     %    	\end{center}
	 	     %    	\caption{Frequency range of the longest protocol}
	 	     %    	\label{rys:long_spectrum}
	 	     %    \end{figure}
Because of the long duration, this type of measurement was performed only for one subject. Total number of repetitions for each ear is ... and it was measured na przestrzeni 1 roku.
\chapter{Data analysis}
Data was analysed in Mathworks MATLAB2015 environment
    \section{Powtarzalność}
    	Rozkład odchylenia std wyników 3 protokołów dla wszystkich badanych
    wąskie biny histogramu. Osobny kolor słupków dla każdego protokołu (DP, SF quick, SF clusters, SF long)
    Osobny rysunek ze wszystkimi częstotliwościami zrzuconymi do jednego worka i tylko 3 kolory słupków.
    
    Różnica między liczbą zaliczonych częstości dla wszystkich badanych bez wyjęcia sondy vs z wyjęciem sondy (wielokrotne porównania! przy wyjęciu porównać średnią ze średnią?) Histogram?
    \section{Spójność wyników}
	porównanie czy w SF zaliczone czestości takie jak w DP. Pokazanie, że DP zalicza dużo więcej.
% 	
%    \begin{equation}
%    x_i(t) = s(t) + n_i(t),
%    \end{equation}
 
     
\chapter{Results}
% (Fig. \ref{rys:usrednianie}).
    

    
    \section{}
	
    \section{}
    


    \chapter{Discussion}
    
    

%\printbibliography[heading=bibintoc]

\end{document}